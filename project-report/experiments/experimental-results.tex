\documentclass[../main.tex]{subfiles}
\begin{document}

\subsection{Experimental results}
\label{subsec:experimental-results}
The following results were obtained by executing the instances with the model that was presented in \hyperref[sec:model]{Section 2}, implemented in \textit{MiniZinc}.\\
Each instance was executed with \textbf{4 time limits}, to see the evolution over time of the data to collect, and with \textbf{6 search strategies}, to see the variation between them and which one would suit best for this kind of problem.
At first, only the \textbf{number of failures} and the \textbf{value of the unweighted objective function}\footnote{By \textit{unweighted objective function} we mean objective function (16) presented in \hyperref[subsubsec:minimizing-total-distance]{Section 2.4.1}.} were computed.
After collecting this kind of data and having understood which kind of search strategy suited best, we executed again all the instances with the \textbf{weighted objective function}\footnote{That is objective function (18) presented in \hyperref[subsubsec:minimizing-total-distance-number-vehicles]{Section 2.4.2}.}, with \textbf{5 weights combinations}, employing only the search strategy \textbf{domWdeg, random, Luby restart (L = 250), LNS (85\%)}.
For that we computed just problem oriented data, i.e. the \textbf{value of the weighted objective function}, the \textbf{total distance travelled by the vehicles} and the \textbf{number of employed vehicles}.\\
The time limits for which data was collected were:
\begin{itemize}
    \item \textbf{30 seconds};
    \item \textbf{1 minute};
    \item \textbf{2 minutes};
    \item \textbf{5 minutes}.
\end{itemize}
The search strategies for which data was collected were:
\begin{itemize}
    \item \textbf{default}: search strategy automatically chosen by the solver, i.e. Gecode;
    \item \textbf{domWdeg, random}: in the search tree, the variable with the smallest value of domain size divided by its weighted degree (i.e. the number of times the variable has caused a failure in a constraint earlier in the search) is chosen and its value is assigned randomly;
    \item \textbf{domWdeg, random, Luby restart (L = 250)}: like the previous search strategy, but the search is restarted from the top of the search tree after $s_i \cdot L$ search steps are performed, with $s_i$ being the $i$-th number in the Luby sequence (\{1,1,2,1,1,2,4,1,1,\dots\}) and $i$ being the $i$-th restart;
    \item \textbf{domWdeg, random, Luby restart (L = 250), LNS (85\%)}: like the previous search strategy, but the \textbf{Large Neighbourhood Search (LNS)} heuristic is exploited; that means that after finding a feasible solution $S$, in the following solution $S'$ a variable value is fixed with a probability of 85\% and the other variables are relaxed;
    \item \textbf{domWdeg, random, Luby restart (L = 250), LNS (15\%)}: like the previous search strategy, but the probability to fix a variable is now 15\% (we wanted to verify the complementary case);
    \item \textbf{first fail, min}: in the search tree, the variable with the smallest domain size is chosen and its value is assigned to the smallest value in its domain.
\end{itemize}
The weights $\alpha$ and $\beta$ for which data was collected were:
\begin{itemize}
    \item $\alpha = 1.0$ and $\beta = 0.0$;
    \item $\alpha = 0.7$ and $\beta = 0.3$;
    \item $\alpha = 0.5$ and $\beta = 0.5$;
    \item $\alpha = 0.3$ and $\beta = 0.7$;
    \item $\alpha = 0.0$ and $\beta = 1.0$.
\end{itemize}
\textit{Note}: $\alpha$ and $\beta$ in the implementation were multiplied by a factor $10$, since the solver had issues with a floating point objective function. Therefore, the following change was applied: $\alpha, \beta \in \mathbb{Z}_+ \; s.t.\; \alpha + \beta = 10$.
\\
\\
(This space was intentionally left blank.)
\newpage

\subfile{experimental-results/mini-example}
\subfile{experimental-results/example}
\subfile{experimental-results/pr01}
\subfile{experimental-results/pr02}
\subfile{experimental-results/pr03}
\subfile{experimental-results/pr04}
\subfile{experimental-results/pr05}
\subfile{experimental-results/pr06}
\subfile{experimental-results/pr07}
\subfile{experimental-results/pr08}
\subfile{experimental-results/pr09}
\subfile{experimental-results/pr10}
\subfile{experimental-results/pr11}

\end{document}