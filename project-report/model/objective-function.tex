\documentclass[../main.tex]{subfiles}
\begin{document}

\subsection{Objective function}
The objective function, i.e. the function which its value needs to be minimized for solving the \textit{VRP} can be of two kinds:
\begin{enumerate}
    \item the first kind requires to minimize just the total distance travelled by the vehicles;
    \item the second kind requires to minimize the total distance together to the number of vehicles that leave the depot, giving weights to the two.
\end{enumerate}
\subsubsection{Minimizing the total distance}
To minimize the total distance we need to compute it first. The expression that allows computing the total distance is:
\begin{center}
    \begin{equation}
        total\_distance = (\sum_{i = 1, i < next_i}^{\sup{}(\mathbb{D}_N)}{distances_{i, next_i}}) + (\sum_{i = 1, i \geq next_i}^{\sup{}(\mathbb{D}_N)}{distances_{next_i, i}})
    \end{equation}
\end{center}
The objective function $f$ which needs to be minimized is finally:
\begin{center}
    \begin{equation}
        f = total\_distance
    \end{equation}
\end{center}

\subsubsection{Minimizing the total distance and the number of vehicles}
The total distance is computed like it was presented in the previous subsubsection. The total number of vehicles that move is computed as follows:
\begin{center}
    \begin{equation}
        used\_vehicles = nvalue(\{vehicle_1,\dots,vehicle_n\})
    \end{equation}
\end{center}
The function $nvalue$ returns the number of different values in the array passed as argument.
Only the first part of the array $vehicle$, the part that has indexes in $\{1,\dots,n\}$, since they refer to vehicles visiting customers.
It is equivalent to bringing all the elements of an array, putting them inside a mathematical set (no duplicated items) and computing the cardinality.\\
The objective function $f$ which needs to be minimized is therefore:
\begin{center}
    \begin{equation}
        f = \alpha{} \cdot total\_distance + \beta{} \cdot used\_vehicles, \; s.t. \; \alpha{}, \beta{} \in \mathbb{R}_+, \; \alpha{} + \beta{} = 1
    \end{equation}
\end{center}
\end{document}