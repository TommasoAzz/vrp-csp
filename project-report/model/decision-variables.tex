\documentclass[../main.tex]{subfiles}
\begin{document}

\subsection{Decision variables}
After showing the input data and the ``variable" $distances$, we shall present the variables that allow the solution for the model.\\
We first need to define two concepts:
\begin{itemize}
    \item \textit{global route}: the concatenation of all visits of all vehicles to the nodes (customers and the depot), without caring about which vehicle has done a certain visit. The global route must include the visits to all the customers and $2 \cdot NumVehicles$ ``logic visits" to the depot;
    \item \textit{vehicle route}: the concatenation of all the visits performed by a specific vehicle, leaving from the depot and ultimately coming back to it. If a vehicle decides not to move, from a logical point of view there will be a move anyway, going from its own exit point to its own entrance point of the depot, which has $d = 0$.
\end{itemize}
The decision variables are the following:
\begin{itemize}
    \item $next \in \mathbb{D}_N^{|\mathbb{D}_N|}$, represents the nodes visited after a visit, i.e. $next_i = j$ means that in the \textit{global route} the visit performed after visiting $node_i$ is to $node_j$;
    \item $vehicle \in \mathbb{D}_N^{|\mathbb{D}_N|}$, represents the vehicles that perform visits, i.e. $vehicle_i = v$ means that $node_i$ is visited by vehicle $v$ (both in the \textit{global} and in the \textit{vehicle route}).
\end{itemize}
These two would suffice if instances of the problem were small but, for larger instances, they do not. An extra variable for performing \textit{symmetry breaking} is needed:
\begin{itemize}
    \item $visited\_customers \in \{0,1\}^{NumVehicles\times{}n}$, represents which vehicles perform the visits to customers, i.e. $visited\_customers_{v,i} = 1$ means that vehicle $v$ visits customer $i$.
\end{itemize}

\end{document}