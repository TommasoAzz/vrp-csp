\documentclass[../main.tex]{subfiles}
\begin{document}

\subsection{Formal problem definition}
After explaining what the VRP is, let's go further in detail to understand how the problem can be modelled.\\
The input of the problem is:
\begin{itemize}
    \item a set of vehicles, of size $m$, in which all of them:
    \begin{itemize}
        \item must either start and leave from the depot (if they are required to move to deliver goods) or stay in it;
        \item have a maximum capacity of goods that can be delivered in their travel to satisfy customer demands;
    \end{itemize}
    \item a set of customers, of size $n$, in which each of them has:
    \begin{itemize}
        \item a demand quantity of the goods that is delivered by the vehicles;
        \item a location;
    \end{itemize}
    \item a depot, which is a place from which all vehicles begin and end their routes.
\end{itemize}
The output of the problem is:
\begin{itemize}
    \item the vehicle serving each customer;
    \item the route (i.e. all customers visited, in which order) travelled by each vehicle.
\end{itemize}
The goal is:
\begin{itemize}
    \item minimizing the total distance travelled by the vehicles, or;
    \item minimizing the total distance and the number of vehicles (achievable by giving weights to these two aspects).
\end{itemize}

\end{document}