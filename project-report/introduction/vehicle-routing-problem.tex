\documentclass[../main.tex]{subfiles}
\begin{document}

\subsection{Vehicle Routing Problem}
\label{subsec:vehicle-routing-problem}
The problem we will describe in this report is, as the title says, the \textbf{Vehicle Routing Problem}, \textbf{VRP} for short.
VRP is a combinatorial optimization problem in which the goal is to optimize the routes that a fleet of vehicles travels, knowing that each vehicle must start its route
from a depot and end it in a depot, which can be the same. The reason why these vehicles need to travel is to visit customers that demand certain goods to be delivered to them.\\
VRP is a generalization of another well-known problem, the \textbf{Travelling Salesman Problem}, \textbf{TSP} for short, in which the shortest path for visiting a list of places by just one vehicle (the ``salesman") must be found and minimized.
VRP is \textit{NP-hard} and can be solved through \textit{mathematical programming}, such as \textit{Integer Linear Programming} (\textit{ILP}) or \textit{Constraint Programming} (\textit{CP}).
Big instances of the problem can be quite hard to solve, and this poses a series of challenges to overcome, which will be shown in the following sections of this report.\\
Visiting customers by the fleet of vehicles is actually just one part of the problem. There are many variants, that specify different needs:
\begin{itemize}
    \item maximizing the total profit by the vehicles, without the need to visit all customers, given that a visit to a customer allows to achieve a certain profit;
    \item moving items from a pickup place to another;
    \item visiting customers in specified time windows;
    \item visiting all customers, but vehicles have a maximum capacity and therefore they have a limit of customers they can visit;
    \item many more.
\end{itemize}
This report will go in detail on how the VRP with vehicles that have a maximum capacity can be solved with \textit{Constraint Programming}.

\end{document}
