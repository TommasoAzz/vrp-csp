\subsubsection{Weighted objective function, vehicles and total distance}
\label{subsubsec:distance-vehicles-weighted-obj-fun}
After analysing the results for the unweighted objective function, we want to highlight how the values of the distance travelled by the vehicles and the number of vehicles change when weights $\alpha$ and $\beta$ of the weighted objective function are changed.
We present such information in the form of two tables, in which for each dataset and for both values - total distance and used vehicles - we compare them to their respective value obtained with weights $\alpha=10$ and $\beta=0$ and show the difference in percentage.\\
Our goal is to understand which combination of weights could be best suited (on average) for this kind of problem: the best one would be that which maximizes the decrement on the total number of used vehicles while minimizing the increment on the total travelled distance.
{
\renewcommand{\arraystretch}{2}
\begin{longtable}[h]{| c | c | c | c | c | c |}
    \hline
    \textbf{Total distance (change in \%)} & \multicolumn{4}{c}{\textbf{Weights ([$\alpha,\beta$])}} & \\
    \hline
    \textbf{Dataset} & \textbf{[10,0]} & \textbf{[7,3]} & \textbf{[5,5]} & \textbf{[3,7]} & \textbf{[0,10]} \\
    \hline
    \endhead
    example              & 0,00\% &  0,00\% &  0,00\% &   0,00\% &  +27,01\% \\
    \hline
    mini-example         & 0,00\% &  0,00\% &  0,00\% &   0,00\% &   +6,27\% \\
    \hline
    pr01                 & 0,00\% & -0,23\% & -0,23\% &  -5,61\% & +217,40\% \\
    \hline
    pr02                 & 0,00\% & -3,44\% & -1,73\% &  -4,16\% &  +80,26\% \\
    \hline
    pr03                 & 0,00\% & +1,15\% & +1,15\% & +24,79\% &  +18,55\% \\
    \hline
    pr04                 & 0,00\% &  0,00\% &  0,00\% &   0,00\% &   +4,23\% \\
    \hline
    pr05                 & 0,00\% &  0,00\% &  0,00\% &   0,01\% &   +0,87\% \\
    \hline
    pr06                 & 0,00\% &  0,00\% &  0,00\% &   0,00\% &   +0,77\% \\
    \hline
    pr07                 & 0,00\% & -5,81\% & -5,81\% &  -4,88\% & +120,75\% \\
    \hline
    pr08                 & 0,00\% & +0,58\% & +1,30\% &  -0,55\% &  +28,36\% \\
    \hline
    pr09                 & 0,00\% &  0,00\% & -0,07\% &  -0,07\% &   +1,09\% \\
    \hline
    pr10                 & 0,00\% &  0,00\% &  0,00\% &   0,00\% &   +0,02\% \\
    \hline
    pr11                 & 0,00\% & -0,24\% & -0,24\% &  +5,21\% & +192,85\% \\
    \hline
    Average              & 0,00\% & -0,61\% & -0,43\% &  +1,13\% &  +53,72\% \\
    \hline
    Average (only pr$n$) & 0,00\% & -0,73\% & -0,51\% &  +1,34\% &  +60,47\% \\
    \hline
\end{longtable}
}

{
\renewcommand{\arraystretch}{2}
\begin{longtable}[h]{| c | c | c | c | c | c |}
    \hline
    \textbf{Used vehicles (change in \%)} & \multicolumn{4}{c}{\textbf{Weights ([$\alpha,\beta$])}} & \\
    \hline
    \textbf{Dataset} & \textbf{[10,0]} & \textbf{[7,3]} & \textbf{[5,5]} & \textbf{[3,7]} & \textbf{[0,10]} \\
    \hline
    \endhead
    example              & 0,00\% & +50,00\% & +50,00\% & +50,00\% &  +50,00\% \\
    \hline
    mini-example         & 0,00\% & -33,33\% & -33,33\% & -33,33\% &  -33,33\% \\
    \hline
    pr01                 & 0,00\% &   0,00\% &   0,00\% &   0,00\% &    0,00\% \\
    \hline
    pr02                 & 0,00\% &  +8,33\% &  +8,33\% &  -8,33\% &  -41,67\% \\
    \hline
    pr03                 & 0,00\% & -15,79\% & -15,79\% &  +5,26\% &  -47,37\% \\
    \hline
    pr04                 & 0,00\% &   0,00\% &   0,00\% &   0,00\% &    0,00\% \\
    \hline
    pr05                 & 0,00\% &   0,00\% &   0,00\% &   0,00\% &    0,00\% \\
    \hline
    pr06                 & 0,00\% &   0,00\% &   0,00\% &   0,00\% &    0,00\% \\
    \hline
    pr07                 & 0,00\% & +28,57\% & +28,57\% &   0,00\% &  -42,86\% \\
    \hline
    pr08                 & 0,00\% & -10,00\% & -10,00\% & -25,00\% &  -40,00\% \\
    \hline
    pr09                 & 0,00\% &   0,00\% &   0,00\% &   0,00\% &    0,00\% \\
    \hline
    pr10                 & 0,00\% &   0,00\% &   0,00\% &   0,00\% &    0,00\% \\
    \hline
    pr11                 & 0,00\% & -50,00\% & -50,00\% & -33,33\% &  -50,00\% \\
    \hline
    Average              & 0,00\% &  -1,71\% &  -1,71\% &  -3,44\% &  -15,79\% \\
    \hline
    Average (only pr$n$) & 0,00\% &  -3,53\% &  -3,53\% &  -5,58\% &  -20,17\% \\
    \hline
\end{longtable}
}

As the tables show, on average there are two combination of weights that look interesting:
\begin{itemize}
    \item $\alpha=7,\beta=3$, because on average it actually minimizes both the distance and the vehicles, even if we must consider that some instances actually increase the number of used vehicles when asked to minimize them;
    \item $\alpha=3,\beta=7$, because on average it increases just a little the total travelled distance and decreases of approximately 5\% the total number of used vehicles, which is good, but again we must consider the fact that there are many cases in which the total travelled distance increases by a not so small factor (perhaps, that is because the solution was stuck in a local minimum, like it is discussed in \hyperref[subsubsec:failures-unweighted-obj-fun]{Section 4.1.1}).
\end{itemize}