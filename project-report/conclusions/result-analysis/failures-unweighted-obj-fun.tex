\subsubsection{Unweighted objective function and failures}
From the experiments performed on the different datasets with different time limits we can conclude that the best search strategy is \textbf{domWdeg, random, Luby restart (L = 250), LNS (85\%)} because in seven instances it returned the minimum value for the objective functions and the minimum value of failures. However for some instances this search strategy was not the best:
\begin{itemize}
    \item In \textbf{mini-example}, even if the two LNS search strategy found the best value for the objective function, they didn't understand that it was the best value and this caused an increasing in the number of failures. In mini-example every other search strategy found the best solution but the minimum number of failures was reached with the default search strategy. This was probably caused by the low difficult of the instance;
    \item In \textbf{example}, the reasoning is similar with mini-example, but in this case the LNS(85\%) search strategy didn't found the best solution for the instance. This was, probably, caused because the search was locked in a local minimum and fixing the 85\% of the variables didn't allow to escape from it. For this instance the minimum number of failures was reached by \textbf{domWdeg, random} search strategy;
    \item In \textbf{pr03}, even if we reached the minimum number of failures with \textbf{domWdeg, random, Luby restart (L = 250), LNS (85\%)}, 223 in five minutes, the minimum solution was found with \textbf{domWdeg, random, Luby restart (L = 250)}. However the difference between the minimum solution found with domWdeg, random, Luby restart (L = 250) and with domWdeg, random, Luby restart (L = 250), LNS (85\%) is only of 2.066.670 (1,90\% of difference). This was probably caused because the LNS search strategy couldn't exit from a local minimum, maybe, increasing the time limit or reducing the percentage of fixed variables the LNS search strategy could have performed better;
    \item In \textbf{pr04} we can say similar things said for pr03 because the minimum number of failures was reached by \textbf{domWdeg, random, Luby restart (L = 250), LNS (85\%)} but the minimum objective function value was reached by \textbf{domWdeg, random, Luby restart (L = 250), LNS (15\%)}. Also in this case LNS(85\%) was stuck in a local minimum from which is possible to exit reducing the percentage of fixed variables, as LNS(15\%) demonstrates;
    \item Also in \textbf{pr05} the minimum number of failures was reached by \textbf{domWdeg, random, Luby restart (L = 250), LNS (85\%)} but the minimum value of the objective function was found using \textbf{first fail, min}. The difference between the two search strategy is 4.439.400 (3,06\% of difference);
    \item In \textbf{pr10} the minimum number of failures was reached by \textbf{domWdeg, random, Luby restart (L = 250), LNS (85\%)} but the minimum value of the objective function was found using \textbf{domWdeg, random, Luby restart (L = 250)}. Also in this case the LNS(85\%) search strategy was stuck in a local minimum, in fact, if we don't fix any variables we can achieve a better value for the objective function;
    \item In \textbf{pr11} we have a different situation: the minimum value for the objective function was reached with \textbf{domWdeg, random, Luby restart (L = 250), LNS (85\%)} but, after five minutes, the minimum number of failures was performed by \textbf{domWdeg, random, Luby restart (L = 250)}. However LNS(85\%) search strategy surpass the number of failures of the other search strategy  only in the five minutes test. Maybe, like in the mini-example and example instances, we found a solution really close at the best solution so the number of failures started to increasing because it couldn't find a better one. 
\end{itemize}
