\subsubsection{Unweighted objective function and failures}
\label{subsubsec:failures-unweighted-obj-fun}
From the experiments performed on the different datasets and time limits, it looks like the best search strategy would be \textbf{domWdeg, random, Luby restart (L = 250), LNS (85\%)}, since in 7/13 instances it returned the minimum value for both the objective function and the number of failures.\\
However, for some instances this search strategy was not the best:
\begin{itemize}
    \item in \textbf{mini-example}, even if the LNS search strategy found the best value for the objective function, it couldn't prove it and this caused a huge increment in the number of failures. In this dataset, every other search strategy found the optimal solution but the minimum number of failures was reached with the default search strategy. This was probably caused by the low complexity of the instance;
    \item in \textbf{example}, something similar to mini-example happened, but in this case the search strategy with LNS (85\%) didn't find the best solution for the instance. This was probably caused because the search was trapped into a local minimum and the 85\% of variable fixing didn't allow to get out from it. For this instance the minimum number of failures was reached by \textbf{domWdeg, random} search strategy;
    \item in \textbf{pr03}, even if we reached the minimum number of failures with \textbf{domWdeg, random, Luby restart (L = 250), LNS (85\%)}, just 223 in 5 minutes, the minimum solution was found with \textbf{domWdeg, random, Luby restart (L = 250)}. Despite this, the difference between the two solutions is just 2.066.670 (1,90\% of difference). Perhaps this was again caused by the search trapped into a local minimum, and maybe increasing the time limit or reducing the percentage of fixed variables in the LNS would have performed better;
    \item in \textbf{pr04} we notice that something similar to pr03 happened. The minimum number of failures was reached by \textbf{domWdeg, random, Luby restart (L = 250), LNS (85\%)} but the minimum objective function value was instead reached by \textbf{domWdeg, random, Luby restart (L = 250), LNS (15\%)}. Even in this case, the former search strategy was stuck in a local minimum from which is possible to escape reducing the percentage of fixed variables, as LNS (15\%) shows;
    \item in \textbf{pr05} the minimum number of failures was reached by \textbf{domWdeg, random, Luby restart (L = 250), LNS (85\%)} but the minimum value of the objective function was found using \textbf{first fail, min}. The difference between the two search strategy is 4.439.400 (3,06\% of difference);
    \item in \textbf{pr10} the minimum number of failures was reached by \textbf{domWdeg, random, Luby restart (L = 250), LNS (85\%)} but the minimum value of the objective function was found using \textbf{domWdeg, random, Luby restart (L = 250)}. Even in this case the LNS(85\%) search strategy was stuck into a local minimum. In fact, without fixing any variable we can achieve a better value for the objective function;
    \item in \textbf{pr11} we have a different situation: the minimum value for the objective function was reached with \textbf{domWdeg, random, Luby restart (L = 250), LNS (85\%)} but, after five minutes, the minimum number of failures was performed by \textbf{domWdeg, random, Luby restart (L = 250)}. However, the search strategy with LNS (85\%) surpassed the number of failures of the other search strategy only in the longer test. This could present us a similar situation to that which happened in mini-example and example, in which we had found solutions really close to the best ones so the number of failures started to increase because it couldn't find something better. 
\end{itemize}
